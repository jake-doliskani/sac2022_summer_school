\documentclass[11pt]{article}

\usepackage[margin = 1in]{geometry}
\usepackage{graphicx}              
\usepackage{amsmath}               
\usepackage{amsfonts}              
\usepackage{amsthm}                
\usepackage{amssymb}
\usepackage{mathrsfs}
\usepackage{url}
\usepackage{xcolor}
\usepackage{hyperref}
\usepackage{algpseudocode}
\usepackage[plain]{algorithm}
\usepackage{enumitem}
\usepackage{authblk}
\usepackage{tcolorbox}
\usepackage{mathtools}
\usepackage{bm}
\usepackage{dsfont}
\usepackage{tikz}
\usetikzlibrary{cd}
%% \usepackage[T1]{fontenc}
%% \usepackage{libertine}
%% \usepackage[libertine]{newtxmath}


\hypersetup{
	unicode = true,
	colorlinks = true,
	citecolor = blue,
	filecolor = blue,
	linkcolor = blue,
	urlcolor = blue,
	pdfstartview = {FitH},
}

% theorem environments
\theoremstyle{plain}
\newtheorem{theorem}{Theorem}[section]
\newtheorem{lemma}[theorem]{Lemma}
\newtheorem{corollary}[theorem]{Corollary}
\newtheorem{proposition}[theorem]{Proposition}
\newtheorem{claim}[theorem]{Claim}
\theoremstyle{definition}
\newtheorem{definition}[theorem]{Definition}
\newtheorem{conjecture}[theorem]{Conjecture}
\newtheorem{example}[theorem]{Example}
\newtheorem{algo-thm}{Algorithm}
\newtheorem*{remark}{Remark}
\newtheorem{note}{Note}
\newtheorem*{problem}{Problem}
\newtheorem*{fact}{Fact}



\algrenewcommand{\Require}{\item[\textit{Input:}]}
\algrenewcommand{\Ensure}{\item[\textit{Output:}]}
\algrenewcommand{\alglinenumber}[1]{#1.}

\newcommand{\tildO}{\tilde{O}}

% roman numerals
\newcommand{\romnum}[1]{\romannumeral #1}
\newcommand{\Romnum}[1]{\uppercase\expandafter{\romannumeral #1}}

\newcommand{\todo}[1]{\textcolor{red}{TODO: #1}}

\DeclareMathOperator{\fieldchar}{char} % characteristic of a field
\DeclareMathOperator{\groupofend}{End} % endomorphism ring
\DeclareMathOperator{\tr}{Tr} % finite field trace
\DeclareMathOperator{\gal}{Gal} % Galois group
\DeclareMathOperator{\order}{ord} % order of an element
\DeclareMathOperator{\lcm}{lcm} % least common multiple
\DeclareMathOperator{\divisor}{div} % divisor on a curve
\DeclareMathOperator{\supp}{supp} % support of a divisor
\DeclareMathOperator{\norm}{N} % norm
\DeclareMathOperator{\negl}{negl} % norm
\DeclareMathOperator{\res}{Res}
\DeclareMathOperator{\GL}{GL}
\DeclareMathOperator{\aut}{Aut}
\DeclareMathOperator{\minpoly}{minpoly}
\DeclareMathOperator{\poly}{poly}
\DeclareMathOperator{\rev}{rev}
\DeclareMathOperator{\entpy}{H}
\let\hom\relax
\DeclareMathOperator{\hom}{Hom}
\DeclareMathOperator{\qft}{F}
\DeclareMathOperator{\E}{\mathbb{E}}
\DeclareMathOperator{\elli}{Ell}


\DeclarePairedDelimiter{\abs}{\lvert}{\rvert}
\DeclarePairedDelimiter{\ket}{\lvert}{\rangle}
\DeclarePairedDelimiter{\bra}{\langle}{\rvert}
\DeclarePairedDelimiterX{\braket}[2]{\langle}{\rangle}{#1 \delimsize\vert #2}
\DeclarePairedDelimiter{\lrang}{\langle}{\rangle}
\DeclarePairedDelimiter{\opnorm}{\lVert}{\rVert}


\def\Q{\mathbb{Q}}
\def\C{\mathbb{C}}
\def\K{\mathbb{K}}
\def\N{\mathbb{N}}
\def\R{\mathbb{R}}
\def\Z{\mathbb{Z}}
\def\F{\mathbb{F}}
\def\P{\mathbb{P}}
\def\MM{\mathsf{M}}
\def\CC{\mathsf{C}}
\def\X{\mathcal{X}}
\def\SX{\mathcal{S(X)}}
\def\U{\mathcal{U}}
\def\PosSemi{\mathrm{Pos}}
\newcommand{\kex}{\mathsf{K}}
\renewcommand{\O}{\mathcal{O}}
\newcommand{\clo}{\text{Cl}(\mathcal{O})}


\title{Quantum Computing And Isogeny-Based Cryptography}

\author{
    SAC 2022 - summer school \\
	Jake Doliskani
}

\date{}
\setlength{\parindent}{0pt}
\sloppy
\allowdisplaybreaks



\begin{document}
\maketitle




We follow \cite{watrous2018theory} for our quantum computing notations. A qubit is usually written as
\[ \alpha\ket{0} + \beta\ket{1} \in \C^2 \]
where $\abs{\alpha}^2 + \abs{\beta}^2 = 1$. For the state of an $n$-qubit quantum system we write
\[ \sum_{x \in \{0, 1\}^n} \alpha_x \ket{x} \in \C^{2^n} \]
where $\sum_x \abs{\alpha_x}^2 = 1$. In the above, we have used the basis $\{ \ket{x} : x \in \{0, 1\}^n \}$ of $\C^{2^n} = (\C^2)^{\otimes n}$, called the \textit{computational basis}. We usually talk about a quantum register. The classical state  of a register $\mathsf{X}$ is represented by a finite alphabet $\Sigma$. The complex Euclidean space associated with $\mathsf{X}$ is denoted by $\X = \C^\Sigma$. Therefore, when we use the computational basis, we are using the alphabet $\Sigma = \{0, 1\}^2$ of binary strings of length $n$. When we are working more than one registers, say $(\mathsf{X}_1, \dots, \mathsf{X}_k)$, with alphabets $(\Sigma_1, \dots, \Sigma_k)$, the state space is $\X_1 \otimes \cdots \otimes \X_k$ where $\X_j = \C^{\Sigma_j}$, i.e., the tensor product of the individual spaces.  



\section{Fourier Transform}

One the most used operations in quantum computing is the quantum Fourier transform (QFT). For an integer $N > 0$, define $\omega_N = e^{2\pi i / N}$. QFT over the group $\Z_N$ is defined by
\[ \mathrm{QFT}_N: \ket{x} \mapsto \frac{1}{\sqrt{N}} \sum_{y \in \Z_N} \omega_N^{xy} \ket{y}. \]
The above is defined for a basis element $\ket{x}$ but it can be extended to the whole space by linearity, for a general superposition we have
\[ \mathrm{QFT}_N: \sum_{x \in \Z_N} \alpha_x \ket{x} \mapsto \sum_{y \in \Z_N} \widehat{\alpha}(y) \ket{y} \]
where $\widehat{\alpha}$ is the classical Fourier transform of the function $\alpha: \Z_N \rightarrow \C$ defined by $\alpha(x) = \alpha_x$. QFT over $\Z_N$ can be done in $\poly(\log N)$ operations.

The Fourier transform over a finite abelian group $G$ is usually stated using the \textit{linear representations} of $G$. Since $G$ is abelian, a representation of $G$ is a group homomorphism $\rho: G \rightarrow \C$. In this case, the representations of $G$ can be identified with the \textit{characters} of $G$. See Serre's book \cite{serre1977linear} for an introduction to representation theory. We denote by $\widehat{G}$ the set of all (inequivalent) characters of $G$. The Fourier transform of a function $f: G \rightarrow \C$ at a character $\chi \in \widehat{G}$ is defined by
\[ \widehat{f}(\chi) = \frac{1}{\sqrt{\abs{G}}} \sum_{x \in G} f(x)\chi(x). \]
The Fourier transform of $f$ is the vector $(\widehat{f}(\chi))_{\chi \in \widehat{G}}$. This is a unitary operation between vector spaces of dimension $\abs{G}$. The quantum Fourier transform over $G$ is defined by
\[ \mathrm{QFT}_G: \ket{x} \mapsto \frac{1}{\sqrt{\abs{G}}} \sum_{\chi \in \widehat{G}} \chi(x) \ket{\chi}. \]

Any finite abelian group $G$ can be written as a product of groups like $\Z_N$. If we write $G = \Z_{N_1} \times \cdots \times \Z_{N_k}$ then QFT over $G$ is
\[ \mathrm{QFT}_G = \mathrm{QFT}_{N_1} \otimes \cdots \otimes \mathrm{QFT}_{N_k}.  \]
Therefore, we can efficiently, i.e., in time $\poly(\log \abs{G})$, compute QFT over finite abelian groups. For 
any $\bm{x} = (x_1, \dots, x_k) \in G$, the function $\chi_{\bm{x}}: G \rightarrow \C$ define by
\begin{equation}
\label{equ:char}
	\chi_{\bm{x}}(\bm{y}) = e^{2\pi i \left( \frac{x_1y_1}{N_1} + \cdots + \frac{x_ky_k}{N_1} 
	\right)}
\end{equation}
is a character of $G$, and the set $\{ \chi_{\bm{x}}: \bm{x} \in G \}$ is a complete set of 
characters of $G$. We can then explicitly write the quantum Fourier transform of $x$ as
\[ \mathrm{QFT}_G \ket{x_1, \dots, x_k} = \sum_{\bm{y} \in G} \chi_{\bm{y}}(\bm{x}) \ket{\bm{y}} = 
\sum_{(y_1, \dots, y_k) \in G} e^{2\pi i \left( \frac{x_1y_1}{N_1} + \cdots + \frac{x_ky_k}{N_1} 
\right)} \ket{y_1, \dots, y_k}. \]



\section{The Hidden Subgroup Problem}

Let $f: G \rightarrow S$ be an efficiently computable function from a finite group $G$ to a finite 
set $S$. Then $f$ is said to hide a subgroup $H \le G$ if the following holds: $f(x) = f(y)$ if and 
only if $xy^{-1} \in H$ for all $x, y \in G$. The hidden subgroup problem (HSP) is to compute a 
generating set for $H$ given access to $f$.

An algorithm for HSP is said to be efficient if it runs in $\poly(\log \abs{G})$ time and memory. 
Although no efficient algorithm is known for the general HSP, there are efficient quantum algorithms 
for the abelian HSP, i.e., when $G$ is abelian \cite{hallgren2003hidden, kaye2007introduction}. The abelian HSP can be completely solved using a 
technique called weak Fourier sampling. To have a closer look at the runtime of the HSP algorithm, 
we briefly discuss it here.

\begin{algo-thm}[Weak Fourier sampling]
	\label{alg:wfs}
	Given a function $f: G \rightarrow S$ that hides a subgroup $H \le G$, sample a	character $\chi \in \widehat{G}$ such that $H \le \ker\chi$.
	\begin{enumerate}[topsep = 0pt, itemsep = 0pt, parsep = 0pt]
		\item Create a superposition over the group $G$ and query $f$ in an ancilla register to 
		obtain the state $\frac{1}{\sqrt{\abs{G}}} \sum_{x \in G} \ket{x, f(x)}$.
		\item Measure the second register. Since $f$ hides $H$, the resulting state will be the  
		superposition $\ket{\psi} := \frac{1}{\sqrt{\abs{H}}} \sum_{h \in H} \ket{ch, f(ch)}$ 
		over a coset $cH$ for some uniformly random $c \in G$.
		\item Apply the quantum Fourier transform to the state $\ket{\psi}$ to obtain
		\[ \ket{\phi} := \frac{1}{\sqrt{\abs{G}\abs{H}}} \sum_{\chi \in \widehat{G}} \sum_{h \in H} 
		\chi(ch) \ket{\chi}. \]
		\item\label{step:msur-rep} Measure the state $\ket{\phi}$ to obtain a character $\chi$.
	\end{enumerate}
\end{algo-thm}

It is not hard to show that the subgroup $H$ is contained in the kernel of the representation 
obtained in Step \ref{step:msur-rep}; see \cite{hallgren2003hidden}. Note that, when $G$ is abelian, 
a representation of $G$ is a homomorphism $\rho: G \rightarrow \C^\times$, and $\ker \rho = \{ g \in 
G \mid \rho(g) = 1 \}$. 

\begin{algo-thm}[Abelian HSP]
	\label{alg:ahsp}
	Given a function $f: G \rightarrow S$ that hides a subgroup $H \le G$, compute a set of 
	generators for $H$.
	\begin{enumerate}[topsep = 0pt, itemsep = 0pt, parsep = 0pt]
		\item Set $s := 4\log \abs{G}$
		\item\label{step:hsp:loop} For $i = 1, \dots, s$, perform a weak Fourier sampling to obtain 
		a representation $\rho_i$.
		\item\label{step:ker} Output a set of generators for $K = \bigcap_{i = 1}^s \ker \rho_i$.
	\end{enumerate}
\end{algo-thm}

\begin{theorem}[{\cite[Theorem 4.3]{hallgren2003hidden}}]
	Algorithm \ref{alg:ahsp} returns a set of generators for $H$ with probability at least $1 - 
	2e^{-\log \abs{G} / 8}$.
\end{theorem}

\begin{theorem}
	\label{thm:ahsp-time}
	Algorithm \ref{alg:ahsp} requires $O(\log \abs{G})$ evaluations of the function $f$ and 
	$O(\log^3 \abs{G})$ elementary operations.
\end{theorem}
\begin{proof}
	Step \ref{step:hsp:loop} of the algorithm performs $4\log \abs{G}$ weak Fourier samplings. Each 
	weak Fourier sampling is done at the cost of one application of the quantum Fourier transform 
	$F_G$ and one query to the function $f$. The cost of applying $F_G$ is $O(\log \abs{G} \log\log 
	\abs{G})$ elementary operations \cite{cleve1998quantum}. The total cost of Step 
	\ref{step:hsp:loop} is then $O(\log \abs{G})$ queries to $f$ and $O(\log^2 \abs{G} \log\log 
	\abs{G})$ elementary operations. To compute the cost of Step \ref{step:ker}, note that given 
	$\rho \in \hat{G}$ we have $\ker \rho = \ker \chi_{\bm{a}}$ for a unique $\bm{a} \in G$, where  
	$\chi_{\bm{a}}$ is define as in \eqref{equ:char}, so we have $\ker \rho = \{ \bm{x} \in G: 
	\frac{a_1x_1}{N_1} + \cdots + \frac{a_kx_k}{N_k} = 0 \bmod 1 \}$. Let $A$ be a matrix with rows 
	of the form $(\frac{a_1}{N_1}, \dots, \frac{a_k}{N_k})$ for all such $\bm{a}$. Then a set of 
	generators for $K$ is the same as a set of generators of the kernel of $A$, which can be 
	obtained using $O(\log^3 \abs{G})$ elementary operations.
\end{proof}




\section{Hard homogeneous spaces}
\label{sec:hhs}

Let $G$ be a finite abelian group. A set $X$ equipped with an action of $G$ is called a $G$-space. 
The action of $G$ on $X$ is transitive if for any $x_1, x_2 \in X$ there exists a $g \in 
G$ such that $gx_1 = x_2$. The action is free if for any $x \in X$ and $g_1, g_2 \in G$ such that 
$g_1x = g_2x$ we have $g_1 = g_2$. 
\begin{definition}
	A homogeneous space for $G$, or a $G$-torsor, is a $G$-space $X$ on which $G$ acts freely and 
	transitively.
\end{definition}
Equivalently, a set $X$ is a homogeneous space for $G$ if for any given pair $x_1, x_2 \in X$ there 
exists a unique $g \in G$ such that $gx_1 = x_2$. In \cite{couveignes2006hard}, Couveignes defines a 
\textit{hard homogeneous space} as follows.
\begin{definition}
\label{def:hhs}
	A $G$-torsor $X$ is called a hard homogeneous space if $G$ and $X$ satisfy certain 
	computational properties. The following operations are required to be easy:
	\begin{itemize}[topsep = 0pt, itemsep = 0pt, parsep = 0pt]
		\item \textit{Group operations}. Given $g_1, g_2 \in G$, compute $g_1^{-1}$ and $g_1g_2$ 
		and decide if $g_1 = g_2$.
		\item \textit{Sampling}. Sample random elements from $G$ with distribution close to uniform.
		\item \textit{Membership}. Given (a string) $x$, decide if $x$ represents an element of $X$.
		\item \textit{Equality}. Given $x_1, x_2 \in X$, decide if $x_1 = x_2$.
		\item \textit{Action}. Given $g \in G$ and $x \in X$, compute $gx \in X$.
	\end{itemize}
	The following problems are required to be hard:
	\begin{itemize}[topsep = 0pt, itemsep = 0pt, parsep = 0pt]
		\item \textit{Vectorization}. Given $x_1, x_2 \in X$, find $g \in G$ such that $gx_1 = x_2$.
		\item \textit{Parallelization}. Given $x_1, x_2, x_3 \in X$ such that $gx_1 = x_2$ for some 
		$g \in G$, find $x_4 = gx_3$.
	\end{itemize}
\end{definition}
To simplify the complexity statements, we shall refer to the first set of operations in Definition \ref{def:hhs} as \textit{elementary operations}.

\paragraph{Key exchange.}
The commutativity of the group action leads to a protocol for key exchange that is a natural 
generalization of the classical Diffie-Hellman: let $x_0$ be a fixed public parameter. Alice selects 
a random $a \in G$ as her private key and publishes $ax_0$ as her public key. Bob does the same for 
a random $b \in G$. A shared secret between Alice and Bob can then be computed as $a(bx_0)$ and 
$b(ax_0)$ respectively. The private keys are protected by the hardness of the vectorization problem, 
and the shared secrete is protected by the hardness of the parallelization problem. 

An example of a hard homogeneous space is a set of supersingular elliptic curves defined over a 
finite field $\F_p$ on which the ideal class group of the endomorphism ring of these elliptic curves 
act. The action of the ideal class group is computed using isogenies. We shall discuss this space in 
more details in Section \ref{sec:csidh}. A classical example of a hard homogeneous space is as 
follows. Let $X$ be a cyclic group of order $n$ and let $G = \aut(X)$ be the group of automorphisms 
of $X$. Then an element of $G$ maps a generator of $X$ to another generator, and hence $G \simeq (\Z 
/ n\Z)^*$. The classical Diffie-Hellman key exchange on $X$ is then given by the above scheme.





\section{A Quantum Reduction}

Let $X$ be a hard homogeneous space for a finite abelian group $G$ and let $x_0 \in X$ be fixed. Let 
$g_1, \dots, g_k \in G$ be a set of generators so that we have an isomorphism $G \cong \lrang{g_1} 
\oplus \cdots \oplus \lrang{g_k}$. Such an isomorphism can be efficiently computed using the quantum
algorithm of \cite{cheung2001decomposing}. We fix this isomorphism and assume it is precomputed. If 
$N_i$ is the order of $g_i$, then the above isomorphism induces the isomorphism $\bigoplus_{i = 1}^k 
\Z / N_i \Z \cong G$ defined by $(y_1, \dots, y_k) \mapsto g_1^{y_1} \cdots g_k^{y_k}$.

\begin{definition}
	The key exchange oracle for $(X, x_0)$ is a function $\kex: X \times X \rightarrow X$ defined as 
	follows. Given any pair of distinct elements $x_1, x_2 \in X$, since $X$ is a $G$-torsor, there 
	exist unique $g_1, g_2 \in G$ such that $x_1 = g_1x_0$ and $x_2 = g_2x_0$. Let $\kex(x_1, x_2) = 
	g_1g_2x_0$.
\end{definition}

The restriction that the inputs of $\kex$ must be distinct is based on the practical assumption for 
key exchange protocols where it is assumed that each party has a unique public key. Given an element 
$x = gx_0 \in X$ where $g \in G$ is unknown, we show that there is an efficient quantum algorithm 
for finding $g$ given access to the key exchange oracle $\kex$.

\begin{lemma}
\label{lem:kex-pow}
	Given $gx_0 \in X$ for some (possibly unknown) $g \in G$, the element $g^nx_0$ can be computed 
	using $O(\log n)$ calls to the key exchange oracle $\kex$ and $O(\log n)$ elementary operations.
\end{lemma}
\begin{proof}
	To compute $g^2x_0$, we first compute the action $g_1gx_0 = g_1(gx_0)$ for an element $g_1 \ne 
	1$ of $G$. Then we can call $\kex$ with input $(g_1gx_0, gx_0)$ to obtain $g_1g^2x_0$, and then 
	compute the action $g_1^{-1}(g_1g^2x_0)$. Repeating the same for $g^2x_0$ we can compute 
	$g^{2^i}x_0$ for any integer $i > 0$. This leads to a binary exponentiation method for computing 
	$g^nx_0$ for any integer $n$.
\end{proof}

Let $N = N_1N_2 \cdots N_k$ be the size of the group $G$ and let 
\[ T = \Z / N \Z \oplus \Z / N_1 \Z \oplus \cdots \oplus \Z / N_k \Z. \]
For an unknown $g \in G$, denote by $f$ the composition $ T \rightarrow G \rightarrow X$ where the 
first map is defined by $(y, y_1, \dots, y_k) \mapsto g^y g_1^{y_1} \cdots g_k^{y_k}$ and the second 
is defined by $h \mapsto hx_0$. In other words
\begin{equation}
\label{equ:map-f}
	\begin{array}{lrll}
		f: & T & \longrightarrow & X \\
		& (y, y_1, \dots, y_k) & \longmapsto & g^y g_1^{y_1} \cdots g_k^{y_k} x_0.
	\end{array}
\end{equation}

\begin{proposition}
\label{prop:f-hide}
	Let $g = g_1^{x_1} \cdots g_2^{x_k}$ for some integers $0 \le x_i < N_i$, $i = 1, \dots, k$. 
	Then the map $f$ defined as in \eqref{equ:map-f} hides the cyclic subgroup $K = \lrang{(1, -x_1, 
	\dots, -x_k)} \le T$.
\end{proposition}
\begin{proof}
	Let $\bm{y} = (y, y_1, \dots, y_k)$ and $\bm{z} = (z, z_1, \dots, z_k)$ be elements of $T$ and 
	assume $f(\bm{y}) = f(\bm{z})$. Then
	\begin{align*}
		x_0 
		& = g^{y - z} g_1^{y_1 - z_1} \cdots g_k^{y_k - z_k} x_0 \\
		& = g_1^{x_1(y - z) + y_1 - z_1} \cdots g_k^{x_k(y - z) + y_k - z_k} x_0.
	\end{align*}
	Since the action of $G$ on $X$ is free, we have $x_i(y - z) + y_i - z_i = 0 \bmod N_i$ for all 
	$i = 1, \dots, k$. Therefore, $\bm{y} - \bm{z} = (y - z)(1, -x_1, \dots, -x_k) \in K$. 
	Conversely, suppose $\bm{y} - \bm{z} = t(1, -x_1, \dots, -x_k)$ for some integer $t \ge 0$. Then 
	\begin{align*}
		f(\bm{y}) 
		& = f(\bm{z} + t(1, -x_1, \dots, -x_k)) \\
		& = g^{t + z} g_1^{z_1 - tx_1} \cdots g_k^{z_k - tx_k}x_0 \\
		& = g^z g_1^{z_1} \cdots g_k^{z_k} (g_1^{tx_1} \cdots g_k^{tx_k} g_1^{-tx_1} \cdots 
		g_k^{-tx_k} x_0) \\
		& = f(\bm{z}) \qedhere
	\end{align*} 
\end{proof}

\begin{lemma}
	\label{lem:compute-f}
	The map $f$ defined as in \eqref{equ:map-f} can be computed using $O(\log \abs{G})$ calls to the 
	key exchange oracle $\kex$ and $O(\log \abs{G})$ elementary operations.
\end{lemma}
\begin{proof}
	Given $(y, y_1, \dots, y_k) \in T$, by Lemma \ref{lem:kex-pow}, the element $g^yx_0$ can be 
	computed using $O(\log N) = O(\log \abs{G})$ calls to $\kex$ and $O(\log \abs{G})$ elementary 
	operations. Each of the elements $g_i^{y_i}x_0$ can be computed using $O(\log N_i)$ calls to 
	$\kex$ and $O(\log N_i)$ elementary operations. The cost of computing all the elements $g^yx_0, 
	g_1^{y_1}x_0, \dots, g_k^{y_k}x_0$ is then $O(\log \abs{G} + \sum_{i = 1}^k \log N_i) = O(\log 
	\abs{G})$. Now, the element $g^y g_1^{y_1} \cdots g_k^{y_k}x_0$ can be computed using $O(k) \in 
	O(\log \abs{G})$ calls to $\kex$. 
\end{proof}

Putting all the above together, we obtain the following:
\begin{theorem}
\label{thm:reduc}
	Given any $x_1 \in X$, let $g \in G$ be the unique element such $x_1 = gx_0$. Then there is a 
	polynomial time quantum algorithm that finds $g$ using $O(\log^2 \abs{G})$ calls to the key 
	exchange oracle $\kex$ and $O(\log^3 \abs{G})$ elementary operations.
\end{theorem}
\begin{proof}
	Let $g = g_1^{x_1} \cdots g_2^{x_k}$ for some integers $0 \le x_i < N_i$, $i = 1, \dots, k$. 
	Then, by Proposition \ref{prop:f-hide}, the map $f$ defined as in \eqref{equ:map-f} hides the 
	subgroup $K = \lrang{(1, -x_1, \dots, -x_k)} \le T$. By Theorem \ref{thm:ahsp-time}, a 
	generating set for $K$, and hence the $x_i$'s, can be computed using $O(\log \abs{G})$ 
	evaluations of $f$ and $O(\log^3 \abs{G})$ elementary operations. Since, by Lemma 
	\ref{lem:compute-f}, each evaluations of $f$ takes $O(\log \abs{G})$ calls to $\kex$ and $O(\log 
	\abs{G})$ elementary operations, the theorem follows.
\end{proof}









\section{The commutative isogeny key exchange}
\label{sec:csidh}

In this section, we briefly discuss a special case of Theorem \ref{thm:reduc} for ideal class groups 
and elliptic curves. For a good reference on elliptic curves see \cite{silverman2009arithmetic} or 
\cite{washington2003elliptic}. Let $p$ be a prime and let $q = p^n$ for an integer $n > 0$. Given 
elliptic curves $E, E'$ over $\F_q$, an isogeny $\phi: E \rightarrow E$ is a rational map that is 
also a group homomorphism. For any finite subgroup $H \le E$ there is a separable isogeny $\phi_H: E 
\rightarrow E / H$, with $\ker \phi_H = H$, that is unique up to isomorphism. An endomorphism of $E$ 
is an isogeny from $E$ to itself. We denote by $\groupofend_F(E)$ the endomorphism ring of $E$ define 
over an extension $F / \F_q$, and by $\groupofend(E)$ the endomorphism ring of $E$ defined over the 
algebraic closure of $\F_q$. The elliptic curve $E$ is called ordinary if $\groupofend(E)$ is an 
order in an imaginary quadratic extension; otherwise $E$ is called supersingular.

Let $\O = \groupofend(E)$ be an order in an imaginary quadratic extension $K$ of $\Q$. The set of 
invertible ideals of $\O$ in $K$ modulo the set of principal ideals form an abelian group under 
multiplication called the ideal class group. We denote the class group of $\O$ by $\clo$. The size 
of $\clo$ is called the class number of $\O$. Let $\mathfrak{a} \in \clo$ be an ideal with norm 
coprime to $p$. The $\mathfrak{a}$-torsion subgroup of $E$ is defined as
\[ E[\mathfrak{a}] = \bigcap_{\alpha \in \mathfrak{a}} \ker \alpha. \]
This is a finite subgroup of $E$ and so there is a unique separable isogeny $\phi_{\mathfrak{a}}: E 
\rightarrow E / E[\mathfrak{a}] = E_{\mathfrak{a}}$. One can show that $\groupofend(E_{\mathfrak{a}}) 
= \O$. This induces an action of $\clo$ on the set of elliptic curves with endomorphism ring $\O$, 
where $\mathfrak{a} E = E / E_{\mathfrak{a}}$.
\begin{problem}
\label{prob:isog}
	Given two elliptic curves $E$ and $E'$ over $\F_q$ with the same enodmorphism ring $\O$, find an 
	ideal $\mathfrak{a} \in \clo$ such that $\mathfrak{a} E = E'$.
\end{problem}
If Problem $\ref{prob:isog}$ is hard then we have the following hard homogeneous space.

\paragraph{Elliptic curve HHS.}
Let $\O \subset \Q(\sqrt{-D})$ be an order, where $D$ is a positive integer, and let $\elli_q(\O)$ 
be the set of elliptic curves over $\F_q$ with endomorphism ring $\O$. Then the action
\[
\begin{array}{rll}
	\clo \times \elli_q(\O) & \longrightarrow & \elli_q(\O) \\
	(\mathfrak{a}, E) & \longmapsto & \mathfrak{a} E
\end{array}
\]
is free and transitive. To make the pair $\clo, \elli_q(\O)$ into an HHS, the action $\mathfrak{a} 
E$ needs to be computed efficiently for an arbitrary $\mathfrak{a} \in \clo$. This, as suggested by 
Couveignes \cite{couveignes2006hard}, can be done by representing the elements of $\clo$ as products 
of prime ideals of small norm. More precisely, let $\mathfrak{l}_1, \dots, \mathfrak{l}_n \in \clo$ 
be a set of non-principal prime ideals of small norms $\ell_i$ such that any $\mathfrak{a} \in \clo$ 
can be written as $\mathfrak{a} = \mathfrak{l}_1^{e_1} \cdots \mathfrak{l}_n^{e_n}$ where the $e_i$ 
are integers in the range $[-M, M]$ for a not too large $M$. Then $\mathfrak{a} E$ can be computed 
by composing isogenies of degree $\ell_i$; see \cite{de2018towards} for details.

Using the same idea as in \cite{de2018towards} for computing $\mathfrak{a} E$, Castryck \textit{et 
al.}~\cite{castryck2018csidh} proposed a significantly more efficient key exchange based on a 
different HHS. Instead of ordinary elliptic curves, their scheme, called CSIDH, is based on
supersingular curves $E$ defined over $\F_p$. For such curves, while $\groupofend(E)$ is a maximal 
order in some quaternion algebra, the smaller ring $\groupofend_F{\F_p}(E)$ is an order in 
$\Q(\sqrt{-p})$. The prime $p$ is chosen such that $p = 4 \ell_1 \cdots \ell_n - 1$ where $\ell_1, 
\dots, \ell_n$ are the smallest $n$ primes. For example, to cover an ideal class group of size 
$\approx 2^{256}$ with products of the form $\mathfrak{l}_1^{e_1} \cdots \mathfrak{l}_n^{e_n}$, it 
is heuristically enough to take $n = 74$ and $-5 \le e_i \le 5$. The above choice of $p$ guarantees 
a decomposition $\ell_i \O = \mathfrak{l}_i \bar{\mathfrak{l}}_i$ for $\mathfrak{l}_i = (\ell_i, \pi 
- 1)$ and $\bar{\mathfrak{l}}_i = (\ell_i, \pi + 1)$, where $\pi$ is the Frobenius endomorphism. The 
action of $\mathfrak{l}_i$ (and $\bar{\mathfrak{l}}_i$) can then be computed efficiently using the 
V\'elu formulas.

\begin{theorem}
	The CRS and CSIDH key exchanges are provably secure. More precisely, given access to a key 
	exchange oracle $\kex$ and given two elliptic curves $E$ and $E'$ over $\F_p$ such that
	\begin{itemize}[topsep = 0pt, itemsep = 0pt, parsep = 0pt]
		\item $E, E'$ are ordinary with $\groupofend(E) = \groupofend(E') = \O$, or
		\item $E, E'$ are supersingular with $\groupofend_{\F_p}(E) = \groupofend_{\F_p}(E') = \O$,
	\end{itemize}
	there is an efficient bounded error quantum algorithm $Q$ for finding $\mathfrak{a} \in 
	\O$ such that $E' = \mathfrak{a} E$. The algorithm $\mathsf{A}$ uses $O(\log^2 p)$ calls to the 
	key exchange oracle $\kex$ and $O(\log^3 p)$ elementary operations.
\end{theorem}
\begin{proof}
    We have $\abs{\clo} = O(\sqrt{p} \log p)$.
\end{proof}








\newpage
\bibliographystyle{plain}
\bibliography{references}

\end{document}



